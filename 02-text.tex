\documentclass[10pt]{article}

\usepackage[protrusion=true,expansion=true]{microtype}

\usepackage{tex/widetext}
\usepackage[timestamp]{tex/varsitybluesnotes}
\usepackage{rotating}
\usepackage[colorlinks]{hyperref}
\usepackage{makeidx}
\usepackage{framed}
\usepackage{fancyvrb}
\usepackage{color}
\usepackage{hyperref}
\hypersetup{colorlinks = true, allcolors = blue}
\usepackage{float}
\usepackage{flafter}
\usepackage{tabularx}
\usepackage{booktabs}
\usepackage{longtable}
\usepackage{multirow}

\usepackage{hyperxmp}
\usepackage[type={CC}, modifier={by-nc-sa}, version={4.0}]{doclicense}
\usepackage{sansmath}
\renewcommand{\familydefault}{\sfdefault}
\sansmath

  \usepackage{amsmath}
  \usepackage{amssymb}
  \usepackage{amsthm}

  \usepackage{setspace}
  \setstretch{1}

% texlive 2020
\newlength{\cslhangindent}
\setlength{\cslhangindent}{1.5em}
\newenvironment{CSLReferences}[2]
  {}
  {\par}

\makeindex

\providecommand{\tightlist}{\setlength{\itemsep}{0pt}\setlength{\parskip}{0pt}}

% To pass between YAML and LaTeX the dollar signs are added by CII
% Syntax highlighting #22

\author{Mauricio Vargas Sepulveda}
\title{The Impact of the Free Trade Agreements Chile-USA and
Chile-China}
\date{}

\setlength\parindent{0pt}
\setlength\parskip{2pt}

% Change the university
\newcommand{\university}{}
% Change the department
\newcommand{\department}{}

\begin{document}

\maketitle

\thispagestyle{empty}
\doclicenseThis
\LARGE{Draft document, NOT peer-reviewed}\normalsize
\tableofcontents
\setcounter{page}{0}
\clearpage

\hypertarget{abstract}{%
\section{Abstract}\label{abstract}}

\hypertarget{purpose}{%
\subsection{Purpose}\label{purpose}}

The purpose of this post is to offer the author's view about the
benefits of subscribing Trade Agreements such as TPP-11.

\hypertarget{approach}{%
\subsection{Approach}\label{approach}}

This article is based on a counterfactual simulation of a `what if'
scenario where the Chile-China and Chile-USA FTAs never existed and how
that political situation would have affected Chile's exports and GDP.

\hypertarget{findings}{%
\subsection{Findings}\label{findings}}

The article shows two estimation methods that reveal the positive effect
of the FTAs on the Chilean economy compared to a counterfactual scenario
of no FTAs.

\hypertarget{introduction}{%
\section{Introduction}\label{introduction}}

This work proposes an econometric experiment to investigate the general
equilibrium effects of FTAs, particularly the effects of the Chile-China
and Chile-USA FTAs. The main reference for the steps discussed here is
Yotov et al. (2016).

The choice of these agreements is because China and the USA are the
No.~1 and No.~2 trading partners with Chile, according to the most
recent United Nations trade data for 2021 (United Nations Statistics
Division 2022).

Focusing on these FTAs, whose effects are misunderstood in public
opinion, primarily due to extreme political positions in Latin America,
aims to measure the observed effects of trade on Chile's GDP.

This econometric experiment relies on panel data to identify the effects
of FTAs and capture the impact of trade costs by using paired fixed
effects (Yotov et al. 2016). The application implements a two-stage
procedure to recover missing bilateral trade costs (Anderson and Yotov
2016).

This work solves a General Equilibrium problem, which considers the
interaction of different market forces similar to the case of multiple
degrees of freedom spring-mass-damper system (Mas-Colell et al. 1995).

The essential ingredient to solve this system is the Poisson Pseudo
Maximum Likelihood estimation (Silva and Tenreyro 2006), implemented as
a constrained regression with an external trade cost vector, resulting
in General Equilibrium PPML.

\hypertarget{estimating-the-impact-of-ftas}{%
\section{Estimating the Impact of
FTAs}\label{estimating-the-impact-of-ftas}}

The analysis begins by specifying the following panel version of the
empirical gravity model in order to obtain estimates of bilateral trade
costs, including an estimate of the average effects of all FTAs (Yotov
et al. 2016):

\begin{equation}
X_{ij,t} = \exp\left[\beta_1 FTA_{ij,t} + \pi_{i,t} + \chi_{j,t} + \mu_{ij}\right] \times \varepsilon_{ij,t}.\label{eq:1}
\end{equation}

Where \(FTA_{ij,t}\) is an indicator variable equal to one if two
countries are members of the same FTA at year \(t\), and zero otherwise.

Following the recommendations in Yotov et al. (2016), this specification
is estimated with the PPML estimator using panel data with exporter-time
fixed effects (\(\pi_{i,t}\)) and importer-time fixed effects
(\(\chi_{j,t}\)).

Pair fixed effects (\(\mu_{ij}\)) are included to alleviate potential
endogeneity concerns of the FTA variable and to control for
time-invariant trade costs at the bilateral level.

Here we evaluate the hypothetical scenario of removing FTAs after they
entered into force in 2004 (USA) and 2006 (China).

\hypertarget{step-i-solve-the-baseline-gravity-model}{%
\subsection{Step I: Solve the baseline gravity
model}\label{step-i-solve-the-baseline-gravity-model}}

We need to estimate the baseline gravity model \eqref{eq:1} to obtain
point estimates of the effect of FTAs and the pair fixed effects and
constructing the bilateral trade costs matrix required to compute the
baseline indexes of interest.

It would be ideal to construct the complete matrix of bilateral trade
costs in order to perform sound counterfactual analysis, but it's often
not possible to identify the complete set of pair fixed effects
estimates in the gravity model when trade flows data are missing (or
zero) for a given pair of countries in the period under investigation.

Following Anderson and Yotov (2016), the two-stage procedure is
implemented in order to recover the complete set of bilateral trade
costs.

We keep the panel dimension of the dataset to identify the effects of
FTAs and comprehensively capture the impact of all time-invariant trade
costs with the use of pair fixed effects (Vargas Sepulveda 2021).

\hypertarget{stage-1-obtain-the-estimates-of-pair-fixed-effects-and-the-effects-of-ftas}{%
\subsubsection{Stage 1: Obtain the estimates of pair fixed effects and
the effects of
FTAs}\label{stage-1-obtain-the-estimates-of-pair-fixed-effects-and-the-effects-of-ftas}}

For estimation purposes, the panel shall contain the years 1990 to 2006
in intervals of four years.

The first stage consists of estimating equation \eqref{eq:1} in order to
obtain the estimates of the bilateral fixed effects for country-pairs
with non-missing (or non-zero) trade flows:

\begin{equation}
X_{ij,t} = \exp\left[0.41 FTA_{ij,t} + \hat{\pi}_{i,t} + \hat{\chi}_{j,t} + \hat{\mu}_{ij}\right].\label{eq:2}
\end{equation}

The estimate of \(\beta_1\) implies that, on average, the FTAs have led
to
\([\exp(\hat{\beta_1}) - 1] \times 100 \approx [\exp(0.41 - 1] \times 100 \approx 51\)
percent increase in trade among members.

\hypertarget{stage-2-regress-the-estimates-of-pair-fixed-effects-on-gravity-variables-and-country-fixed-effects}{%
\subsubsection{Stage 2: Regress the estimates of pair fixed effects on
gravity variables and country fixed
effects}\label{stage-2-regress-the-estimates-of-pair-fixed-effects-on-gravity-variables-and-country-fixed-effects}}

The second stage involves using the estimates of the pair fixed effects
(\(\mu_{ij}\)) from equation \eqref{eq:2} as the dependent variable in a
regression where the covariates include the set of standard gravity
variables along with importer and exporter fixed effects (Larch and
Yotov 2016):

\begin{equation}
t_{ij}^{1-\sigma} = \exp[\mu_{ij}] = \exp\left[\beta_1 DIST_{ij} + \beta_2 CNTG_{ij} + \beta_3 LANG_{ij} + \beta_3 CLNY_{ij} + \hat{\pi}_{i,t} + \hat{\chi}_{j,t}\right]. \label{eq:3}
\end{equation}

The predictions from regression \eqref{eq:3} are used to fill in the
missing trade cost values in order to obtain the complete set of
bilateral trade costs \(\hat{t}_{ij}^{1-\sigma}\).

Once the full vector of bilateral trade costs is obtained, we include it
as a constraint in the baseline gravity specification \eqref{eq:2},
which will return estimates for the importer and exporter fixed effects.
The result is consistent with the trade cost vector and can be used to
recover the corresponding values of the multilateral resistances.

A similar constrained estimation procedure should be performed when the
trade cost vector is obtained externally, including when it is
constructed with a calibration method.

\hypertarget{step-ii-define-a-counterfactual-scenario}{%
\subsection{Step II: Define a counterfactual
scenario}\label{step-ii-define-a-counterfactual-scenario}}

We need to define the hypothetical removal of the USA and China FTAs
with Chile. By re-defining the FTA dummy variable, \(FTA^{CFL}_{ij,t}\),
as if FTAs were not in place by setting the original FTA indicator
variable to be equal to zero for trade between Chile and the USA after
2004 and Chile and China after 2006.

\hypertarget{step-iii-solve-the-counterfactual-model}{%
\subsection{Step III: Solve the counterfactual
model}\label{step-iii-solve-the-counterfactual-model}}

We need to construct the counterfactual indexes of interest in the
Conditional GE and the Full Endowment GE scenarios of removing the
abovementioned FTAs.

The Conditional GE effects from the removal of the FTAs above are
computed by re-estimating the econometric gravity specification
\eqref{eq:1} for 2004 and 2007, the year of entry into force of the
agreements, subject to constraints reflecting the counterfactual
scenario:

\begin{equation}
X_{ij} = \exp\left[\hat{\beta}_1 FTA^{CFL}_{ij} + \pi^{CFL}_{i} + \chi^{CFL}_{j} + \hat{t}^{1-\sigma}_{ij}\right]\times \varepsilon^{CFL}_{ij}. \label{eq:4}
\end{equation}

Equation \eqref{eq:4} is estimated under the constraints that the
Chile-USA and Chile-China FTAs never existed (\(FTA^{CFL}_{ij}\)) and
the coefficient of the FTA dummy and the bilateral fixed effects are
equal to their baseline values, \(\hat{\beta}_1\) and
\(\hat{t}^{1-\sigma}_{ij}\) respectively, such that no part of the trade
costs besides the FTA dummy changes.

The PPML estimates of the directional effects from equation \eqref{eq:4}
can be used to recover the conditional multilateral resistance indexes
\(\pi^{CFL}_{i}\) and \(\chi^{CFL}_{j}\) subject to normalization.

The values of the Full Endowment GE effects of the removal of the FTAs
are directly obtained by implementing an iterative procedure, which
sequentially allows for endogenous factory-gate prices, followed by
income, expenditure and trade to adjust to the counterfactual shock.

\hypertarget{step-iv-collect-construct-and-report-indexes-of-interest}{%
\subsection{Step IV: Collect, construct, and report indexes of
interest}\label{step-iv-collect-construct-and-report-indexes-of-interest}}

Table 1 reports the results of the counterfactual analysis, including
the percentage difference between the baseline values and their Full
Endowment counterparts of the main variables of interest for each
country in the sample.

\hypertarget{simulation-results}{%
\section{Simulation results}\label{simulation-results}}

Despite some specifications and data limitations, the results presented
and discussed above are comparable with findings from existing related
studies. GEPPML produces a theoretically sound model.

\begin{longtable}[t]{lrrrrrr}
\caption{Conditional GE and Full Endownment GE simulation results}\\
\toprule
% exporter & change\_xi\_cfl & change\_xi\_full & change\_rgdp\_full & change\_imr\_full & change\_omr\_full & change\_p\_i\_full\\
Exporter & \multicolumn{1}{c}{Conditional GE} & \multicolumn{5}{c}{Full Endownment GE}\\
 & \multicolumn{1}{c}{$\%\Delta$ Trade} & $\%\Delta$ Trade & $\%\Delta$ rGDP & $\%\Delta$ IMR & $\%\Delta$ OMR & $\%\Delta$ Prices\\
\midrule
% 0-deu & 0.00 & -0.01 & 0.00 & 0.00 & 0.00 & 0.00\\
deu$^*$ & 0.00 & -0.01 & 0.00 & 0.00 & 0.00 & 0.00\\
arg & -0.10 & -0.07 & -0.01 & 0.02 & -0.02 & 0.01\\
aus & -0.02 & -0.02 & 0.00 & 0.00 & 0.00 & 0.00\\
aut & 0.00 & 0.00 & 0.00 & 0.00 & 0.00 & 0.00\\
bel & 0.00 & 0.00 & 0.00 & 0.00 & 0.00 & 0.00\\
\addlinespace
bgr & 0.00 & 0.00 & 0.00 & 0.00 & 0.00 & 0.00\\
bol & -0.39 & -0.31 & -0.01 & 0.07 & -0.06 & 0.05\\
bra & -0.05 & -0.04 & 0.00 & 0.01 & -0.01 & 0.01\\
can & -0.02 & -0.03 & 0.00 & -0.01 & 0.01 & -0.01\\
che & 0.00 & 0.00 & 0.00 & 0.00 & 0.00 & 0.00\\
\addlinespace
% chl & 9.30 & 9.87 & 0.55 & -0.06 & -0.57 & 0.49\\
\textbf{chl} & \textbf{9.30} & \textbf{9.87} & \textbf{0.55} & \textbf{-0.06} & \textbf{-0.57} & \textbf{0.49}\\
chn & 0.09 & 0.10 & 0.00 & 0.00 & -0.01 & 0.01\\
cmr & -0.01 & -0.01 & 0.00 & 0.00 & 0.00 & 0.00\\
col & -0.07 & -0.06 & 0.00 & 0.01 & -0.01 & 0.01\\
cri & -0.03 & -0.03 & 0.00 & 0.00 & 0.00 & 0.00\\
\addlinespace
cyp & 0.00 & 0.00 & 0.00 & 0.00 & 0.00 & 0.00\\
dnk & 0.00 & 0.00 & 0.00 & 0.00 & 0.00 & 0.00\\
ecu & -0.13 & -0.10 & 0.00 & 0.02 & -0.02 & 0.02\\
egy & -0.01 & -0.01 & 0.00 & 0.00 & 0.00 & 0.00\\
esp & -0.01 & -0.01 & 0.00 & 0.00 & 0.00 & 0.00\\
\addlinespace
fin & -0.01 & -0.01 & 0.00 & 0.00 & 0.00 & 0.00\\
fra & -0.01 & -0.01 & 0.00 & 0.00 & 0.00 & 0.00\\
gbr & -0.01 & -0.01 & 0.00 & 0.00 & 0.00 & 0.00\\
grc & -0.01 & -0.01 & 0.00 & 0.00 & 0.00 & 0.00\\
hkg & -0.01 & -0.01 & 0.00 & 0.00 & 0.00 & 0.00\\
\addlinespace
hun & 0.00 & 0.00 & 0.00 & 0.00 & 0.00 & 0.00\\
idn & -0.01 & -0.01 & 0.00 & 0.00 & 0.00 & 0.00\\
ind & -0.01 & -0.01 & 0.00 & 0.00 & 0.00 & 0.00\\
irl & 0.00 & 0.00 & 0.00 & 0.00 & 0.00 & 0.00\\
irn & 0.00 & 0.00 & 0.00 & 0.00 & 0.00 & 0.00\\
\addlinespace
isl & -0.01 & -0.01 & 0.00 & 0.00 & 0.00 & 0.00\\
isr & -0.01 & -0.01 & 0.00 & 0.00 & 0.00 & 0.00\\
ita & -0.01 & -0.01 & 0.00 & 0.00 & 0.00 & 0.00\\
jor & -0.01 & -0.01 & 0.00 & 0.00 & 0.00 & 0.00\\
jpn & -0.01 & -0.01 & 0.00 & 0.00 & 0.00 & 0.00\\
\addlinespace
ken & 0.00 & -0.01 & 0.00 & 0.00 & 0.00 & 0.00\\
kor & -0.02 & -0.02 & 0.00 & 0.00 & 0.00 & 0.00\\
kwt & -0.01 & -0.01 & 0.00 & 0.00 & 0.00 & 0.00\\
lka & -0.01 & -0.01 & 0.00 & 0.00 & 0.00 & 0.00\\
mac & -0.01 & -0.01 & 0.00 & 0.00 & 0.00 & 0.00\\
\addlinespace
mar & 0.00 & 0.00 & 0.00 & 0.00 & 0.00 & 0.00\\
mex & -0.02 & -0.03 & 0.00 & 0.00 & 0.01 & -0.01\\
mlt & 0.00 & 0.00 & 0.00 & 0.00 & 0.00 & 0.00\\
mmr & -0.01 & -0.01 & 0.00 & 0.00 & 0.00 & 0.00\\
mus & 0.00 & 0.00 & 0.00 & 0.00 & 0.00 & 0.00\\
\addlinespace
mwi & 0.00 & 0.00 & 0.00 & 0.00 & 0.00 & 0.00\\
mys & -0.01 & -0.01 & 0.00 & 0.00 & 0.00 & 0.00\\
ner & 0.00 & 0.00 & 0.00 & 0.00 & 0.00 & 0.00\\
nga & -0.01 & -0.01 & 0.00 & 0.00 & 0.00 & 0.00\\
nld & -0.01 & -0.01 & 0.00 & 0.00 & 0.00 & 0.00\\
\addlinespace
nor & 0.00 & -0.01 & 0.00 & 0.00 & 0.00 & 0.00\\
npl & -0.01 & -0.01 & 0.00 & 0.00 & 0.00 & 0.00\\
pan & -0.01 & -0.01 & 0.00 & 0.00 & 0.00 & 0.00\\
phl & -0.01 & -0.01 & 0.00 & 0.00 & 0.00 & 0.00\\
pol & 0.00 & 0.00 & 0.00 & 0.00 & 0.00 & 0.00\\
\addlinespace
prt & 0.00 & 0.00 & 0.00 & 0.00 & 0.00 & 0.00\\
qat & -0.01 & -0.01 & 0.00 & 0.00 & 0.00 & 0.00\\
rom & 0.00 & 0.00 & 0.00 & 0.00 & 0.00 & 0.00\\
sen & 0.00 & 0.00 & 0.00 & 0.00 & 0.00 & 0.00\\
sgp & 0.00 & -0.01 & 0.00 & 0.00 & 0.00 & 0.00\\
\addlinespace
swe & 0.00 & 0.00 & 0.00 & 0.00 & 0.00 & 0.00\\
tha & -0.01 & -0.01 & 0.00 & 0.00 & 0.00 & 0.00\\
tto & -0.01 & -0.02 & 0.00 & -0.01 & 0.01 & -0.01\\
tun & 0.00 & 0.00 & 0.00 & 0.00 & 0.00 & 0.00\\
tur & -0.01 & -0.01 & 0.00 & 0.00 & 0.00 & 0.00\\
\addlinespace
tza & 0.00 & 0.00 & 0.00 & 0.00 & 0.00 & 0.00\\
ury & -0.06 & -0.04 & 0.00 & 0.01 & -0.01 & 0.01\\
usa & 0.24 & 0.23 & 0.01 & -0.01 & 0.01 & -0.01\\
zaf & -0.01 & -0.01 & 0.00 & 0.00 & 0.00 & 0.00\\
\bottomrule
\end{longtable}

\hypertarget{concluding-remarks}{%
\section{Concluding remarks}\label{concluding-remarks}}

Column I of Table 1 reveals that, under the conditional general
equilibrium scenario, the member countries of the FTAs above experience
the largest export increase, ranging from \(0.09\) (China) to \(9.30\)
(Chile), percent of their total exports.

Conditional effects, unlike direct (or partial) effects, account for
trade diversion and direct effects are often an upper bound for our
estimation (Yotov et al. 2016). Part of the increase in trade with
member countries comes at the expense of trade with non-members because
of fixed productive factors in the short run.

Trade diversion also explains why the Conditional GE effects on non-FTA
countries' exports are adverse, albeit small (less than 0.5 percent for
all countries).

Non-member countries facing the most significant conditional decrease in
exports appear to be geographically close to the three economies, while
the countries experiencing the most negligible conditional effect tend
to be countries with weak trade ties, at least to Chile, China and the
USA.

Column II of Table 1 reveals that the values of the Full Endowment
general equilibrium effects of the FTAs above on exports are
qualitatively identical to the corresponding conditional equilibrium
effects, despite some quantitative differences.

The Full Endowment GE effects on the FTA members are slightly larger
(i.e., \(9.87\%\) versus \(9.30\%\) for Chile) suggesting that part of
the decrease in bilateral trade costs due to the creation of the FTAs
translates into additional gains for the producers in the member
countries who enjoy higher producer prices.

In most cases, the increase in the size of FTA members mitigates the
adverse effects on non-members' exports.

Column III of Table 1, the counterfactual analysis, suggests that the
Full Endowment welfare effects of FTAs are favourable for its members,
ranging between \(0\) and \(0.55\) percent, and slightly negative or
null for non-member countries.

In our model, an increase of \(0.55\%\) over Chile's GDP comes from the
FTAs, which means Chile's welfare improved after the FTAs.

The decomposition of the Full Endowment GE effects reported in columns
IV, V, and VI suggests that both consumers and producers in member
countries of the FTAs face positive effects with lower inward
multilateral resistances (for consumers) and lower outward multilateral
resistances, which translate into higher factory-gate prices (for
producers) relative to the effects on the consumers in the reference
country (i.e., Germany for the current case).

\hypertarget{codes}{%
\section{Codes}\label{codes}}

All the codes for the simulation are available at

\url{https://github.com/pachadotdev/general-equilibrium-ftas-chile}.

\hypertarget{references}{%
\section*{References}\label{references}}
\addcontentsline{toc}{section}{References}

\hypertarget{refs}{}
\begin{CSLReferences}{1}{0}
\leavevmode\vadjust pre{\hypertarget{ref-anderson2016terms}{}}%
Anderson, James E, and Yoto V Yotov. 2016. {``Terms of Trade and Global
Efficiency Effects of Free Trade Agreements, 1990--2002.''}
\emph{Journal of International Economics} 99: 279--98.

\leavevmode\vadjust pre{\hypertarget{ref-larch2016general}{}}%
Larch, Mario, and Yoto V Yotov. 2016. {``General Equilibrium Trade
Policy Analysis with Structural Gravity.''}

\leavevmode\vadjust pre{\hypertarget{ref-mas1995microeconomic}{}}%
Mas-Colell, Andreu, Michael Dennis Whinston, Jerry R Green, et al. 1995.
\emph{Microeconomic Theory}. Vol. 1. Oxford university press New York.

\leavevmode\vadjust pre{\hypertarget{ref-silva2006log}{}}%
Silva, JMC Santos, and Silvana Tenreyro. 2006. {``The Log of Gravity.''}
\emph{The Review of Economics and Statistics} 88 (4): 641--58.

\leavevmode\vadjust pre{\hypertarget{ref-uncomtrade}{}}%
United Nations Statistics Division. 2022. {``UN COMTRADE. International
Merchandise Trade Statistics.''} United Nations.

\leavevmode\vadjust pre{\hypertarget{ref-vargas2021solutions}{}}%
Vargas Sepulveda, Mauricio. 2021. \emph{Solutions Manual for an Advanced
Guide to Trade Policy Analysis in r}. United Nations.

\leavevmode\vadjust pre{\hypertarget{ref-yotov2016advanced}{}}%
Yotov, Yoto V, Roberta Piermartini, Monteiro Jose Antonio, and Mario
Larch. 2016. \emph{An Advanced Guide to Trade Policy
Analysis\hspace{0pt}: The Structural Gravity Model}. WTO iLibrary.

\end{CSLReferences}

% Index?
% \printindex

\end{document}
